\documentclass[12pt,a4paper]{article}

\usepackage[utf8]{inputenc}
\usepackage[T2A]{fontenc}
\usepackage[russian]{babel}
\usepackage{amsfonts,amsmath,amssymb,longtable,hhline}
\usepackage{multicol}
\usepackage{color,soul}
\usepackage{booktabs, tabularx}
\usepackage{graphicx}
\usepackage{titlesec}

\hoffset=-1.5cm\voffset=-2cm \textwidth=17.1cm \textheight=23.0cm
\parindent=1.20cm 
\righthyphenmin=2     
\renewcommand*{\baselinestretch}{1.05} 
\tolerance=1000         

\renewcommand{\thefootnote}{}

\begin{document}
\large

\noindent УДК 004.94; 656.1; 519.8
\bigskip

\centerline{\bf \LARGE Перколяционный анализ транспортной сети }
\centerline{\bf \LARGE с выявлением критических дорог.}
\bigskip

\centerline{\bf © 2025 г. \ \ Е. С. Данович$^{1,*}$, \ \ В. А. Ивенин$^{1}$, \ \ О. О. Рыбалов$^{1}$, \ \ М. С. Титов$^{1}$}

\bigskip
\centerline{$^{1}$ Центральный университет, 123056, г. Москва, ул. Гашека, д. 7, стр. 1}


\bigskip

% дописать аннотация

\medskip
\textit{Ключевые слова:} перколяция, транспортные сети, критические дороги, графовые модели, визуализация.
\medskip

\section{Введение}\label{Introduction}

Транспортные сети представляют собой сложные инфраструктурные системы, подверженные постоянному риску сбоев, вызванных как естественными причинами 
(перегрузки в часы пик, дорожно-транспортные происшествия, перекрытия), так и чрезвычайными ситуациями 
(стихийные бедствия, отказы ключевых инфраструктурных элементов, таких как кольцевые развязки или тоннели). 
Определение наиболее критичных участков сети, таких как мосты или ключевые перекрестки, и прогнозирование момента, когда сбои в работе отдельных, 
даже территориально небольших, участков сети могут приводить к нелинейным, каскадным эффектам, 
вызывающим масштабные заторы и паралич транспортного сообщения на обширных территориях, является актуальной задачей. 
Так, 11 декабря 2025 года ввиду перекрытия нескольких улиц в центре Москвы по данным сервиса «Яндекс.Карты» движение встало: 
10-бальные пробки по всей Москве \cite{rbc2025}(рис.~\ref{fig:moscow_traffic}). 

\begin{figure}[!h]
    \centering
    \includegraphics[width=0.6\textwidth]{./view/10red.png}
    \caption{Транспортная ситуация в Москве 11 декабря 2025 года по данным сервиса <<Яндекс.Карты>>}
    \label{fig:moscow_traffic}
\end{figure}

В качестве решения проблемы перегрузки центра, как на данном примере, 
было предложено перенести железнодорожные вокзалы за пределы МКАД \cite{gluharuv1995, gluharuv2013}. 
Такое решение уменьшило бы общий автомобильный трафик в пределах города, особенно в центральном транспортном узле, 
и позволило бы использовать освободившиеся территории для уменьшения плотности автомобилей, и с их помощью можно было бы избежать такой проблемы.

Проблема определения критических элементов и порогов устойчивости транспортных сетей не является новой и активно исследуется в научной литературе 
с применением различных подходов, включая теорию сложных сетей и перколяционный анализ, доказательство возможности применения которого доказано в \cite{nekrasova2015}. 
В дальнейшем теория перколяции была успешно применена для нахождения усовершенствованного подхода к выявлению критических рёбер в транспортных сетях \cite{gasparyan2025, balamirzoev2019}. 
В последующих научных исследованиях ученые приходят к выводу, что для эффективного моделирования потоков, 
а также для нахождения наиболее критичных узлов сети целесообразно сначала провести глубокий структурный анализ всей сети, указав наиболее влиятельные компоненты, 
и только после этого выделять зону влияния каждого из них и рассматривать их отдельно друг от друга \cite{khabarov2024}. Изучение идеи учета динамики сети показывает, 
что простого топологического анализа недостаточно. Результаты исследования \cite{gasparyan2025} свидетельствуют о том, 
что учет пространственно-временной сети является важным условием для точного прогнозирования критических рёбер в реальном или конкретно заданном времени. 
Задача автоматизированного планирования сети лесных дорог \cite{pyatin2020} была решена в \cite{katarov2023} с использованием графовой математической модели, 
что позволяет применить данный метод для аналогичного рассмотрения транспортной сети в городе.

Несмотря на наличие значительного объема исследований, посвященных отдельным аспектам анализа транспортных систем, 
остается не до конца решенной задача комплексного применения теории перколяции для точного определения 
момента наступления момента потери связности транспортной сети в реальных городских сетях, 
подверженных отказам ключевых инфраструктурных элементов (мостов, перекрестков). Существующие работы часто фокусируются либо на абстрактных моделях, 
либо на качественной оценке критичности. В данной статье мы стремимся заполнить этот пробел. Целью нашего исследования является разработка 
и апробация количественной методологии, которая позволит не только точно найти критические дороги и мосты, оказывающие наибольшее влияние на пропускную способность системы, 
но и определить порог перколяции -- ту критическую долю отключенных дорог, при которой транспортная сеть необратимо теряет связность 
и перестает функционировать как единое целое.

\section{Модели и методы}

Для реализации поставленной цели необходима формулировка конкретных исследовательских задач и выбор верного научного аппарата. 
В данном исследовании транспортная инфраструктура разделяется на 3 различные начальные карты для Москвы целиком, 
для Северного Административного округа и для Замоскворечья и каждая моделируется как ориентированный граф $G=(V,E)$, 
где узлы V представляют собой ключевые транспортные развязки и перекрестки, а рёбра E соответствуют соединяющим их дорожным участкам. 
Каждой карте присваиваются атрибуты: Количество узлов, количество ребер и количество O-D пар. 
Где набор O-D (Отправление-Назначение) пар находится путем выбора несколько точек в "центре" (условном центре, в зависимости от графа: 
для Москвы от центра города на 8 км радиус это примерно ТТК, для САО чуть больше 8 км берем, для Замоскворечья поменьше, 
в итоге чтобы в этом условном центральном районе было намного меньше точек чем на "периферии"), и несколько точек на "периферии". 
Далее создаем n пар между этими точками, где n=5000 для Москвы, n=3000 для САО и n=1000 для Замоскворечья, чтобы каждая пара была "центр"\-"периферия". 
Таким образом мы будем смотреть на важность дорог, примерно моделируя утренний час пик когда все едут в центр. 
Мы предлагаем метод перехода от статичной структурной схемы к функциональной динамической модели, 
используя инструментарий теории перколяции для симуляции процесса постепенного выведения элементов сети из строя. 
Центральным элементом предлагаемой методологии является моделирование случайной и целенаправленной деградации сети с постепенным ухудшением пропускной способности 
для последующего анализа функциональной устойчивости системы. Для каждой из них будут считаться метрики LLC --- процентное 
соотношение количества существующих путей от любой точки до любой другой к количеству всех путей из всех точек и Efficiency --- доля
увеличения среднего времени пути при удалении каждого отдельного ребра. Где она рассчитывается формулой:
\[ Efficiency = \frac{1}{n(n-1)} \sum_{i \neq j} \frac{1}{d_{ij}} \]

где:
\begin{itemize}
    \item $n$ --- общее число узлов
    \item $d_{ij}$ --- кратчайшее время в пути между узлами $i$ и $j$. Если пути нет, то $1/d_{ij} = 0$
\end{itemize}

Это позволяет установить критический порог перколяции ($p_c$). 
Данный показатель представляет собой точное значение пропорции нефункциональных элементов инфраструктуры, при достижении которого происходит нелинейный фазовый переход: 
единая транспортная система фрагментируется на изолированные сегменты, что приводит к нарушению ключевых транспортных потоков и системному коллапсу.

Для реализации исследования использовался комплекс методов теории перколяции и современных программных инструментов: библиотеки OSMnx, NetworkX, Folium и Kepler.gl. 
Актуальные географические данные загружались с помощью OSMnx, что позволило получить реалистичную модель дорожной сети города, 
включая исследуемые мосты и перекрестки. Анализ и расчеты: основные вычисления и реализация алгоритмов проводились на базе библиотеки NetworkX. 
Этот инструмент использовался для: постепенного удаления ребер, расчета метрик связности сети, определения критического порога перколяции ($p_c$), 
при котором сеть теряет функциональность. Визуализация результатов: для наглядного представления 
и интерпретации пространственных данных применялись инструменты интерактивной визуализации: 
Folium использовался для создания карт с выделением критических участков; 
Kepler.gl обеспечил возможность построения сложных многослойных геопространственных визуализаций сценариев отказов и распространения заторов.

\section{Результаты и обсуждение}


% Итого



\section{Заключение}


% Итого


\begin{thebibliography}{99}

\bibitem{nekrasova2015}
Некрасова А. А., Соколов С. С. Исследование возможности применения теории перколяции для управления потоками данных в информационных сетях на транспорте // Сборник научных трудов Государственного университета морского и речного флота имени адмирала С. О. Макарова. 2015. № 35. С. 138–146.

\bibitem{khabarov2024}
Хабаров В. И., Беков М. А., Квашнин В. Е. Критические объекты транспортной инфраструктуры мегаполисов и агломераций // Вестник Сибирского государственного университета путей сообщения. 2024. № 3 (70). С. 20–27.

\bibitem{gasparyan2025}
Гаспарян Г. А. Обнаружение критически важных звеньев в пространственно-временных маршрутных сетях с использованием теории сложных сетей // Crede Experto: транспорт, общество, образование, язык. 2025. № 3. С. 131–142.

\bibitem{rbc2025}
URL: [https://www.rbc.ru/society/11/12/2025/693ae9839a79476666be9d4c](https://www.rbc.ru/society/11/12/2025/693ae9839a79476666be9d4c)

\bibitem{pyatin2020}
Пятин Д. С. Совершенствование методов решения задачи автоматизированного планирования сети лесных дорог // Современные проблемы лесного хозяйства : материалы Всероссийской научно‑практической конференции. Санкт‑Петербург, 2020. С. 127–131.

\bibitem{katarov2023}
Катаров В. К., Рожин Д. В., Сюнёв В. С. Оптимальное проектирование сети лесных дорог: от методов к решениям // Resources and Technology. 2023. Т. 20, № 3. С. 32–47.

\bibitem{balamirzoev2019}
Баламирзоев А. Г., Батманов Э. З., Султанахмедов М. А., Муртузов М. М., Игитов Ш. М. Управление транспортными потоками на основе перколяционной стохастической модели // Информационные технологии в науке и образовании : материалы Всероссийской научно‑практической конференции. Махачкала : АЛЕФ, 2019. С. 13–16.


\bibitem{gluharuv1995}
Глухарев К. К., Вишнев И. П., Исаков А. В., Фролов К. В. Устройство транспортной системы и способ регулирования транспортно-пассажирским потоком мегаполиса: патент РФ №2104363, приоритет от 24 мая 1995 г.


\bibitem{gluharuv2013}
Глухарев, К. К. О моделировании автомобильных потоков на магистральной сети / К. К. Глухарев, А. М. Валуев, И. Н. Калинин, Н. М. Улюков // Труды Московского физико-технического института. — 2013. — Т. 5, № 4 (20). — С. 102–114.


\end{thebibliography}

\end{document}
