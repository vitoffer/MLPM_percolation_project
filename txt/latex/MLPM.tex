\documentclass[12pt,a4paper]{article}

\usepackage[utf8]{inputenc}
\usepackage[T2A]{fontenc}
\usepackage[russian]{babel}
\usepackage{amsfonts,amsmath,amssymb,longtable,hhline}
\usepackage{multicol}
\usepackage{color,soul}
\usepackage{booktabs, tabularx}
\usepackage{graphicx}

\hoffset=-1.5cm\voffset=-2cm \textwidth=17.1cm \textheight=23.0cm
\parindent=1.20cm 
\righthyphenmin=2     
\renewcommand*{\baselinestretch}{1.05} 
\tolerance=1000         

\renewcommand{\thefootnote}{}

\begin{document}
\large

\noindent УДК 004.94; 656.1; 519.8
\bigskip

\centerline{\bf Перколяционный анализ транспортной сети }
\centerline{\bf с выявлением критических дорог.}
\bigskip

\centerline{\bf © 2025 г. \ \ Е. С. Данович$^{1,*}$, \ \ В. А. Ивенин$^{1}$, \ \ О. О. Рыбалов$^{1}$, \ \ М. С. Титов$^{1}$}

\bigskip
\centerline{$^{1}$ Центральный университет, 123056, г. Москва, ул. Гашека, д. 7, стр. 1}
\centerline{$^{*}$\it email: ivanov@gmail.com} % сюда подставь нужный email

\bigskip

% дописать аннотация

\medskip

\textit{Ключевые слова:} перколяция, транспортные сети, критические дороги, графовые модели, визуализация.
\medskip

\section{Введение}\label{Introduction}

% вставить картинки

Транспортные сети представляют собой сложные инфраструктурные системы, подверженные постоянному риску сбоев, вызванных как естественными причинами (перегрузки в часы пик, дорожно-транспортные происшествия, перекрытия), так и чрезвычайными ситуациями (стихийные бедствия, отказы ключевых инфраструктурных элементов, таких как кольцевые развязки или тоннели). Определение наиболее критичных участков сети, таких как мосты или ключевые перекрестки, и прогнозирование момента, когда сбои в работе отдельных, даже территориально небольших, участков сети могут приводить к нелинейным, каскадным эффектам, вызывающим масштабные заторы и паралич транспортного сообщения на обширных территориях, является актуальной задачей. Так, 11 декабря 2025 года ввиду перекрытия нескольких улиц в центре Москвы по данным сервиса «Яндекс.Карты» движение встало: 10-бальные пробки по всей Москве \cite{rbc2025}. В качестве решения проблемы перегрузки центра, как на данном примере, было предложено перенести железнодорожные вокзалы за пределы МКАД. Такое решение уменьшило бы общий автомобильный трафик в пределах города, особенно в центральном транспортном узле, и позволило бы использовать освободившиеся территории для уменьшения плотности автомобилей, и с их помощью можно было бы избежать такой проблемы.

Проблема определения критических элементов и порогов устойчивости транспортных сетей не является новой и активно исследуется в научной литературе с применением различных подходов, включая теорию сложных сетей и перколяционный анализ, доказательство возможности применения которого приведено в \cite{nekrasova2015}. В дальнейшем теория перколяции была успешно применена для нахождения усовершенствованного подхода к выявлению критических рёбер в транспортных сетях \cite{gasparyan2025,balamirzoev2019}. В последующих исследованиях показано, что для эффективного моделирования потоков, а также для нахождения наиболее критичных узлов сети целесообразно сначала провести глубокий структурный анализ всей сети, указав наиболее влиятельные компоненты, и только после этого выделять зону влияния каждого из них и рассматривать их отдельно \cite{khabarov2024}. Дальнейшее изучение идеи учёта динамики сети показывает, что простого топологического анализа недостаточно: учёт пространственно‑временной структуры является важным условием для точного прогнозирования критических рёбер в реальном или конкретно заданном времени \cite{gasparyan2025}. Задача автоматизированного планирования сети лесных дорог \cite{katarov2023} была решена в \cite{pyatin2020} с использованием графовой математической модели, что позволяет применить аналогичный подход к рассмотрению городской транспортной сети.

Несмотря на значительный объём исследований, посвящённых отдельным аспектам анализа транспортных систем, задача комплексного применения теории перколяции для точного определения момента потери связности реальных городских транспортных сетей, подверженных отказам ключевых инфраструктурных элементов (мостов, перекрёстков), остаётся не до конца решённой. Существующие работы часто фокусируются либо на абстрактных моделях, либо на качественной оценке критичности. Целью настоящего исследования является разработка и апробация количественной методологии, которая позволяет не только находить критические дороги и мосты, оказывающие наибольшее влияние на пропускную способность системы, но и определять порог перколяции — критическую долю отключённых дорог, при которой транспортная сеть необратимо теряет связность и перестаёт функционировать как единое целое.

\section{Постановка задачи}

Для реализации поставленной цели необходима формулировка конкретных исследовательских задач и выбор верного научного аппарата. В данном исследовании транспортная инфраструктура моделируется как ориентированный граф $G=(V,E)$, где узлы V представляют собой ключевые транспортные развязки и перекрестки, а рёбра E соответствуют соединяющим их дорожным участкам и мостам. Каждому ребру присваивается атрибут — пропускная способность $C_e$ и степень текущей загрузки $L_e$. Мы предлагаем метод перехода от статичной структурной схемы к функциональной динамической модели, используя инструментарий теории перколяции для симуляции процесса постепенного выведения элементов сети из строя. Центральным элементом предлагаемой методологии является моделирование случайной и целенаправленной деградации сети с постепенным ухудшением пропускной способности для последующего анализа функциональной устойчивости системы. Это позволяет установить критический порог перколяции ($p_c$). Данный показатель представляет собой точное значение пропорции нефункциональных элементов инфраструктуры, при достижении которого происходит нелинейный фазовый переход: единая транспортная система фрагментируется на изолированные сегменты, что приводит к нарушению ключевых транспортных потоков и системному коллапсу.

\section{Модели и методы}

Для реализации поставленной цели необходима формулировка конкретных исследовательских задач и выбор подходящего математического аппарата. В данном исследовании транспортная инфраструктура моделируется как ориентированный граф \(G=(V,E)\), где множество узлов \(V\) соответствует ключевым транспортным развязкам и перекрёсткам, а множество рёбер \(E\) задаёт соединяющие их дорожные участки и мосты. Каждому ребру присваиваются атрибуты пропускной способности \(C_e\) и степени текущей загрузки \(L_e\). Предлагается переход от статичной структурной схемы к функциональной динамической модели с использованием теории перколяции для имитации постепенного выведения элементов сети из строя. Центральным элементом методологии является моделирование случайной и целенаправленной деградации сети с постепенным ухудшением пропускной способности для анализа функциональной устойчивости системы. Это позволяет установить критический порог перколяции \(p_c\), то есть долю нефункциональных элементов инфраструктуры, при достижении которой происходит фазовый переход: единая транспортная система фрагментируется на изолированные компоненты, что приводит к нарушению основных потоков и системному коллапсу.

В исследовании используется комплекс методов теории перколяции и современных программных инструментов: библиотеки OSMnx, NetworkX, Folium и Kepler.gl. Моделирование сети выполняется на основе данных OSMnx, что обеспечивает реалистичное описание дорожной сети города, включая ключевые мосты и перекрёстки. Библиотека NetworkX применяется для реализации алгоритмов постепенного удаления рёбер, вычисления метрик связности и определения критического порога перколяции \(p_c\). Для пространственной визуализации результатов используются интерактивные инструменты: Folium для построения карт с выделением критических участков и Kepler.gl для многослойных геопространственных визуализаций сценариев отказов и распространения заторов.

\section{Результаты и обсуждение}

% графики и анализ

\section{Заключение}

% итого

\begin{thebibliography}{99}

\bibitem{nekrasova2015}
Некрасова А. А., Соколов С. С. Исследование возможности применения теории перколяции для управления потоками данных в информационных сетях на транспорте // Сборник научных трудов Государственного университета морского и речного флота имени адмирала С. О. Макарова. 2015. № 35. С. 138–146.

\bibitem{khabarov2024}
Хабаров В. И., Беков М. А., Квашнин В. Е. Критические объекты транспортной инфраструктуры мегаполисов и агломераций // Вестник Сибирского государственного университета путей сообщения. 2024. № 3 (70). С. 20–27.

\bibitem{gasparyan2025}
Гаспарян Г. А. Обнаружение критически важных звеньев в пространственно-временных маршрутных сетях с использованием теории сложных сетей // Crede Experto: транспорт, общество, образование, язык. 2025. № 3. С. 131–142.

\bibitem{rbc2025}
URL: https://www.rbc.ru/society/11/12/2025/693ae9839a79476666be9d4c

\bibitem{pyatin2020}
Пятин Д. С. Совершенствование методов решения задачи автоматизированного планирования сети лесных дорог // Современные проблемы лесного хозяйства : материалы Всероссийской научно‑практической конференции. Санкт‑Петербург, 2020. С. 127–131.

\bibitem{katarov2023}
Катаров В. К., Рожин Д. В., Сюнёв В. С. Оптимальное проектирование сети лесных дорог: от методов к решениям // Resources and Technology. 2023. Т. 20, № 3. С. 32–47.

\bibitem{balamirzoev2019}
Баламирзоев А. Г., Батманов Э. З., Султанахмедов М. А., Муртузов М. М., Игитов Ш. М. Управление транспортными потоками на основе перколяционной стохастической модели // Информационные технологии в науке и образовании : материалы Всероссийской научно‑практической конференции. Махачкала : АЛЕФ, 2019. С. 13–16.

\end{thebibliography}

\end{document}
