\documentclass[12pt,a4paper]{article}

\usepackage[utf8]{inputenc}
\usepackage[T2A]{fontenc}
\usepackage[russian]{babel}
\usepackage{amsfonts,amsmath,amssymb,longtable,hhline}
\usepackage{multicol}
\usepackage{color,soul}
\usepackage{booktabs, tabularx}
\usepackage{graphicx}

\hoffset=-1.5cm\voffset=-2cm \textwidth=17.1cm \textheight=23.0cm
\parindent=1.20cm 
\righthyphenmin=2     
\renewcommand*{\baselinestretch}{1.05} 
\tolerance=1000         

\renewcommand{\thefootnote}{}

\begin{document}
\large


\noindent DDD \textit{номер sfsfg}
\bigskip

\centerline{\bf dfgdf СТАТЬИ ПИШЕТСЯ КРУПНО}
\centerline{\bf ВТОРАЯ СТРОЧКА НА СЛУЧАЙ ДЛИННОГО НАЗВАНИЯ}
\centerline{\bf }
\bigskip

\centerline{\bf © 2025 г. \ \ И. И. Иванов$^{1,*}$, \ \ С. С. Сидоров $^{1,2}$}

\bigskip
\centerline{$^{1}$ Центральный университет, 123056, г. Москва, ул. Гашека, д. 7, стр. 1}
\centerline{$^{2}$ Московский физико-технический университет, 141701, Московская область,}
\centerline{г. Долгопрудный, Институтский переулок, д. 9}
\centerline{$^{*}$\it email: ivanov@gmail.com} % указывается почта автора для переписки на случай возникновения вопросов и развития сотрудничества

\bigskip

Это краткая аннотация к твоей работе. Напиши, какую задачу решаешь и какие алгоритмы, методы и модели использованы в работе, а также основные полученные результаты: выявленные закономерности и тому подобное. Обычно объём аннотации должен быть не менее 150 слов.
\medskip

\textit{Ключевые слова:} перколяция, двумерные карты, машинное обучение, математика.
\medskip

\section{Введение}\label{Introduction} % Обязательный раздел

В этом разделе нужно показать актуальность задачи. Для этого перечисли ключевые работы по теме и опиши их результаты. Затем сделай переход к своей работе и расскажи, чем она качественно отличается от предыдущих: методом, подходом или алгоритмом. Нужно показать, какие есть недостатки у существующих методов и подходов и как ты их решаешь в работе.

В конце введения обозначь краткое содержание разделов работы. 

\medskip

\textit{\textbf{Пример}. В первом разделе была поставлена задача работы, представлены и охарактеризованы использованные данные. Во втором разделе описывается, какими методами и инструментами была решена указанная задача. В третьем разделе приводятся и анализируются результаты работы, излагаются выявленные закономерности. Четвёртый раздел посвящён обсуждению результатов и будущим исследованиям в области. В заключительном разделе кратко излагаются результаты работы и полученные закономерности.}

\end{document}
